\chapter{Introduction}
\label{introduction}

\section{Problem Statement}
Languages are different. Linguistic typology studies classification of world languages depending on their structural and functional features. Because of the diversity of different languages, many criteria can be used to classify languages into different groups \citep{wals}. For example, according to subject-verb-object positioning, languages can be grouped into different sets: SOV (such as French, German, Spanish and Chinese), SVO (such as English and Chinese) and so on, where the abbreviation represents the order of subject(S), verb(V) and object(O). Phonologically, patterns in the structure and distributions of sound systems are investigated by linguistics to classify world's languages based on phonological properties. As summarized by \citep{wals}, properties including vowel and consonant inventory, consonant-vowel ratio, syllable structure, rhythm types, etc. are used to represent the difference in phonology across different languages. Some of those properties mainly measure segmental information while others measure supra-segmental information of one language's phonological system. Those phonological properties result in diverse acoustic characteristics when we listen to speech recordings in different languages. Another important outcome of those different phonological properties across languages is that when a speaker speaks in a language other than his mother tongue, the speech he produced will be perceived to have accent, which comes from the interplay of the phonological difference of the first language (L1, the speaker's mother tongue) and second language(L2, the language the speaker is speaking). The study in this dissertation will focus on accented speech.

Accented speech is the result of L2 speech being produced by a sensorimotor control system that has overlearned L1 phonological patterns, including both phoneme sound contrasts and rhythmic composition. According to L1 acquisition theory, the malleability of human brain in terms of language is strongest only for a limited time (sometime between age 5 and puberty), which is referred to as the critical period hypothesis. After that period, language acquisition is much more difficult. The critical period hypothesis from L1 acquisition was extended to include L2 acquisition, positing that a critical age or period after which L2 speech production could not be native like \citep{long1990maturational}. Actually, for L2 learning, findings in neuroscience favored the concept of a ``sensitive'' or ``optimal'' period over a ``critical'' one. Further research revealed that the speech learning model (SLM) accounted for exceptions that can not be explained by the critical period hypothesis. For example, it was reported that both early L2 learners failed to achieve native like production while late L2 learners did \citep{flege1995second}. The SLM also hypothesized a shared phonological space for both L1 and L2 speech sounds and used ``equivalence classification'' to explain why a learner might not create a new phonetic category for an L2 sound perceived as similar to an L1 sound. Basically, SLM emphasized the influence of pre-established L1 phonetic categories on the way L2 sounds are perceived, how it changes over time and also the formation of phonetic categories which is used to produce L2 speech sounds. Later studies reported that for speech prosody acquisition, the influence from the speaker's L1 also play a big role in the formation of phonological patterns to produce L2 speech prosody.

Accentedness is usually used to measure the perceived difference between accented speech produced by L2 learners and speech produced by native speakers. There are multiple ways to define accentedness. A more general definition in literature was proposed in \citep{mccullough2013acoustic} considering previous definitions: accentedness refers to perception of deviations from a pronunciation norm that a listener attributes to the talker not speaking the target language natively. This definition focuses on the difference of foreign accented speech compared to speech produced by native speakers. In second language learning and education practice, accentedness evaluation is very important to designing specific learning targets for different learners based on their level of accentedness, monitoring the learning progress and qualifying or quantifying the learning outcomes. One common experimental design in the study of perceived foreign accents is to have participants rate the degree of accentedness in various auditory stimuli, and then to relate these ratings to properties measured in the stimuli. Much research has been done to study the relationship between perceived foreign accents and acoustic characteristics of accented speech, such as voice onset time (VOT) \citep{major1987english}, word duration, stressed or unstressed vowel duration ratio \citep{shah2002temporal}, formants movement deviation from L1 acoustic values \citep{munro1993productions}, etc. In addition to the largely segmental acoustic properties suggested by the findings of previous studies, some studies focus on supra-segmental information, including prosodic and global temporal properties, of foreign accented speech \citep{munro2010detection,kang2010relative}. Both segmental and supra-segmental acoustic measurements have been shown to be correlated with perceived accentedness.

Though it is clear that the perception of accentedness correlates strongly with how far the phonological patterns of produced accented speech are from patterns of native speech, what has not been studied is whether the distance to the phonological patterns of the speaker's mother tongue matters. According to SLM, phonetic systems of L2 learners respond to L2 sounds by adding new phonetic categories, or modifying existing L1 phonetic categories \citep{flege1995second}. SLM claims that new phonetic categories may be formed for an L2 sound given sufficient dissimilarity from the closest L1 sound; equivalence classification may block the category formation for an L2 sound, thus the original L1 phonetic category will be used to process both L1 and L2 sound, resulting in similar L2 production with L1 sound. Since SLM mainly focuses on the phonetic system, later studies also investigate the acquisition of L2 rhythm patterns \citep{rasier2007prosodic,ordin2015acquisition}. The findings in these studies reveal that while the general trend is moving closer to the L2 rhythm patterns as the accent is milder, there still exist effects of L1 rhythm patterns for speakers from different L1 groups. Based on these observations, we may ask:

\begin{enumerate}
\item How does the distance from the functional phonological system for accented speech to the actual L1 phonological system correlate with the perceived accentedness? Is the position of the phonological patterns of accented speech in the L1 and L2 phonological space a better choice to model the perception of accentedness?
\item Different L1s are at different relative positions with L2 in subspaces of the phonological system, mainly phonetic space and rhythmic space. For example, German and English are both stress-timed languages and thus are closer to each other in rhythmic space compared to French; Mandarin is syllable-timed language and has very different phonetic inventory compared to English so it is far from English in both subspaces. Will these contrastive properties be transferred to the accented speech during L2 acquisition? Furthermore, will those measurements on rhythmic space contribute more to the perception of accentedness compared to measurements on phonetic space for German speakers speaking English?
\end{enumerate}

To answer these questions, the following hypotheses will be tested in this dissertation:
\begin{enumerate}
\item The phonological distance between accented speech and speaker's L1 are negatively correlated with perceived accentedness; If this distance information is added to the feature sets for automatic accentedness evaluation, the performance will be improved.
\item Different L1s are relatively different in terms of the distance to L2 in both phonetic subspace and prosody subspace. Based on this, it can be hypothesized that the phonological properties in different subspaces (phonetic or prosodic) of accented speech produced by speakers from different L1 backgrounds will have different correlational relationship with accentedness score. For example, German is close to English in prosodic subspace but relatively far from English in phonetic subspace. With the hypothesis, it can be predicted that prosodic features of accented speech with German as L1 is less correlated with accentedness score compared to phonetic features. Furthermore, this will result in different relative importance of phonetic and prosodic features across different L1s in regression analysis of acoustic measurements and accentedness score.
\end{enumerate}

The above are the research hypotheses this dissertation will test. Specifically, the current study will explore the relationship between measurements of phonological system with perceived accentedness using a computational model that extracts representative features of both phonetic and rhythmic subspaces. Given that the deviation of accented speech from the target L2 phonological patterns highly correlates with the accentedness score, this study investigates whether the deviation of accented speech from the original L1 phonological patterns has negative correlation with the accentedness score (i.e. the higher the deviation, the milder the accent); whether integrating measurements related to L1 can improve the modeling capability of accentedness perception and whether contrastive patterns of L1s and target L2 can be transferred to L1 accented speech.

\section{Significance of the study}

Billions of people are learning a second (or higher order) language nowadays. The number of people living in a second language environment is also increasing with economic globalization. A good understanding of the process of second language learning and accented speech perception is of great importance to successful speech communication in terms of both education, social science and communication science. The current study investigates the perception of accented speech. It aims to achieve better understanding of how the L1s of second language learners affect the perception of their accentedness phonologically and how the phonetic system and rhythmic patterns contribute to the perception respectively. With a interdisciplinary research methods combining speech learning and perception theories with speech and language technologies, the current study will have impact on both the theoretic development of second language learning and accented speech perception, and the technologies of Computer Assisted Pronunciation Training (CAPT) and Computer Aided Language Learning (CALL).

\section{Outline of the dissertation}

The dissertation is divided into 8 chapters. Chapter 2 introduces the general background of the current study by reviewing several bodies of research on second language learning and accent speech perception theories and practices: including differential analysis of world languages, second language learning theories, acoustic characteristics of accented speech and computational model of accentedness perception. The motivation and predictions are then presented based on the literature review. Chapter 3 describes the methodology employed in this study including data collection, acoustic analysis and experimental design. Chapter 4 investigates the influence of L1's phonetic system on the accented speech perception. Chapter 5 investigates the influence of L1's rhythm patterns on the accented speech perception. Chapter 5 combines information of L1's phonetic and rhythmic patterns to build a computational model for better accentedness perception. Chapter 7 provides a general discussion of the experimental results and tries to extend the current theories on L2 learning and accented speech perception. It also introduced the possible implications to both theoretic and practical studies on accented speech. Chapter 8 concludes the current study.



