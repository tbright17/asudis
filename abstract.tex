\begin{abstract}
Much evidence has shown that first language (L1) plays an important role in the formation of L2 phonological system during second language (L2) learning process. This combines with the fact that different L1s have distinct phonological patterns to indicate the diverse L2 learning outcomes for speakers from different L1 backgrounds. Usually, accented speech is investigated with either segmental (short-term speech measurements on short-segment or phoneme level) or suprasegmental (long-term speech measurements on word, long-segment, or sentence level) measurements. This dissertation hypothesizes that phonological distances between accented speech and speakers' L1 are negatively correlated with perceived accentedness. Moreover, contrastive phonological distinctions between L1s and L2 will manifest themselves in the accented speech produced by speaker from these L1s.

To test the hypotheses, this study comes up with a computational model to analyze the accented speech properties in both segmental and suprasegmental feature space. The core parts of this computational model are feature extraction schemes to extract pronunciation and prosody representation of accented speech based on existing techniques in automatic speech recognition field. Correlation analysis on both segmental and suprasegmental feature space is conducted to look into the L1's influence on accentedness perception across several L1s. Multiple regression analysis is employed to investigate how the L1's effect impacts the perception of foreign accent perception, and how accented speech produced by speakers from different L1s behaves distinctly on segmental and suprasegmental feature spaces. Results unveil the potential application of the methodology in this study to provide quantitative analysis of accented speech and extend current studies in L2 learning theory to large scale. Practically, this study further shows that the computational model proposed in this study can benefit automatic accentedness evaluation system by adding features related to speakers' L1s.
\end{abstract}
