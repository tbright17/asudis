\begin{abstract}
Much evidence has shown that first language (L1) plays an important role in the formation of L2 phonological system during second language (L2) learning process. This combines with the fact that different L1s have distinct phonological patterns to indicate the diverse L2 speech learning outcomes for speakers from different L1 backgrounds. This dissertation hypothesizes that phonological distances between accented speech and speakers' L1 speech are also correlated with perceived accentedness, and the correlations are negative for some phonological properties. Moreover, contrastive phonological distinctions between L1s and L2 will manifest themselves in the accented speech produced by speaker from these L1s. To test the hypotheses, this study comes up with a computational model to analyze the accented speech properties in both segmental (short-term speech measurements on short-segment or phoneme level) and suprasegmental (long-term speech measurements on word, long-segment, or sentence level) feature space. The benefit of using a computational model is that it enables quantitative analysis of L1's effect on accent in terms of different phonological properties. The core parts of this computational model are feature extraction schemes to extract pronunciation and prosody representation of accented speech based on existing techniques in speech processing field. Correlation analysis on both segmental and suprasegmental feature space is conducted to look into the relationship between acoustic measurements related to L1s and perceived accentedness across several L1s. Multiple regression analysis is employed to investigate how the L1's effect impacts the perception of foreign accent, and how accented speech produced by speakers from different L1s behaves distinctly on segmental and suprasegmental feature spaces. Results unveil the potential application of the methodology in this study to provide quantitative analysis of accented speech, and extend current studies in L2 speech learning theory to large scale. Practically, this study further shows that the computational model proposed in this study can benefit automatic accentedness evaluation system by adding features related to speakers' L1s.
\end{abstract}
