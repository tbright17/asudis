\chapter{Consent form for accentedness score collection on AMT}
\label{sec:appendix2}



\subsubsection{Introduction}
The purposes of this form are to provide you (as a prospective research study participant) information that may affect your decision as to whether or not to participate in this research and to record the consent of those who agree to be involved in the study.
\subsubsection{Researchers}

Dr. Julie Liss, a Professor in the Department of Speech \& Hearing Sciences (College of Health Solutions) at ASU, and Dr. Visar Berisha, an Assistant Professor in the Department of Speech \& Hearing Sciences and the School of Electrical, Computer, and Energy Engineering at ASU, have invited your participation in a research study.
\subsubsection{Study purpose}
We are collecting perceived degree of accentedness from people aged 18 and older who have normal hearing. We will use these accentedness ratings to study the impact of non-native English speaker��s native language on the perceived accentedness.

\subsubsection{Description of research study}
If you decide to participate, then you will join a study involving research of the perception of accented speech. Your participation will be completely online and will last no longer than 1 hour. If you agree to participate, we ask that you be seated in a quiet room in front of a computer. You will listen to a paragraph spoken by different individuals in English and asked to give a general impression of the accentedness of each speaker on a 1-4 scale. Research completed based on these accentedness ratings will provide an understanding of the impact of non-native English speaker��s native language on perceived accentedness.

\subsubsection{Risks}
There are no known risks from taking part in this study.


\subsubsection{Benefits}
Although there may be no direct benefits to you, these transcriptions may improve our understanding of accented speech. This may, in turn, allow for the development of computer-aided second language learning system.

\subsubsection{Confidentiality}
All information obtained in this study is strictly confidential. The results of this research study may be used in reports, presentations, and publications, but the researchers will not identify you.

\subsubsection{Withdraw privilege}
Your participation in this project is completely voluntary. There is no penalty for not participating, or for choosing to withdraw from participation at any time. Your decision will in no way affect your relationship with ASU or your grade in any course.
Should you choose to withdraw from the study, your digital audio-video files will not be saved and will be discarded electronically.

\subsubsection{Costs and payments}
The researchers want your decision about participating in the study to be absolutely voluntary. Yet they recognize that your participation may pose some inconvenience. You will receive \$1.5 for your participation, paid via Amazon Mechanical Turk.

\subsubsection{Voluntary consent}
Any questions you have concerning the research study or your participation in the study, before or after your consent, will be answered by Dr. Julie Liss at (480) 965-9136.
If you have questions about your rights as a subject/participant in this research, or if you feel you have been placed at risk; you can contact the Chair of the Human Subjects Institutional Review Board, through the ASU Office of Research Integrity and Assurance, at 480-965 6788.
This form explains the nature, demands, benefits and any risk of the project. By signing this form you agree knowingly to assume any risks involved. Remember, your participation is voluntary. You may choose not to participate or to withdraw your consent and discontinue participation at any time without penalty or loss of benefit. In signing this consent form, you are not waiving any legal claims, rights, or remedies. A copy of this consent form will be offered to you.



By clicking ``Agree'', you consent to participate in the above study and indicated that:
\begin{enumerate}
\item you have read the above information
\item you voluntarily agree to participate
\item you are at least 18 years of age
\end{enumerate}
